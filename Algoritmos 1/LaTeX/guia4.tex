\documentclass[12pt]{article} % Tipo de documento (artículo) y tamaño de fuente (12pt)

% Paquetes necesarios
\usepackage[utf8]{inputenc} % Codificación de caracteres
\usepackage[spanish]{babel} % Idioma español
\usepackage{amsmath,amsfonts,amssymb} % Símbolos matemáticos
\usepackage{graphicx} % Para incluir imágenes
\usepackage{hyperref} % Para enlaces clicables (opcional)

% Información del documento
\title{Título de tu Documento}
\author{Tu Nombre}
\date{\today} % Fecha de hoy

\begin{document}

\maketitle % Crea el título

% Resumen (opcional)
\begin{abstract}
Resumen breve de tu trabajo.
\end{abstract}

% Contenido principal
\section{Introducción}
Aquí comienza tu texto. Puedes dividirlo en secciones y subsecciones.

\section{Metodología}
Describe los métodos utilizados.

\subsection{Análisis de datos}
Explica cómo analizaste los datos.

\section{Resultados}
Presenta tus resultados.

\section{Conclusiones}
Resume tus hallazgos principales.

% Referencias (opcional)
\begin{thebibliography}{9} % 9 es el máximo número de referencias
\bibitem{referencia1} Autor1, Título1, etc.
\bibitem{referencia2} Autor2, Título2, etc.
\end{thebibliography}

\end{document}
