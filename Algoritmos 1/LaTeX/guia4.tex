\documentclass[12pt]{article} % Tipo de documento (artículo) y tamaño de fuente (12pt)

% Paquetes necesarios
\usepackage[utf8]{inputenc} % Codificación de caracteres
\usepackage[spanish]{babel} % Idioma español
\usepackage{amsmath,amsfonts,amssymb} % Símbolos matemáticos
\usepackage{graphicx} % Para incluir imágenes
\usepackage{hyperref} % Para enlaces clicables (opcional)

\title{Introduccion a los Algoritmos}
\author{Agustin Mascó}
\date{\today} 

\begin{document}

\maketitle 


\begin{abstract}
Ejercicios prácticos de la materia Introduccion a los Algoritmos
\end{abstract}

\pagebreak

\section{Práctico 1}
Aquí comienza tu texto. Puedes dividirlo en secciones y subsecciones.

\section{Práctico 2}
Describe los métodos utilizados.

\subsection{Práctico 3}
Explica cómo analizaste los datos.

\section{Práctico 4}

\begin{enumerate}
    \item Realizados en FaMAF/Algoritmos 1/Haskell/guia4.hs
    \item Realizados en FaMAF/Algoritmos 1/Haskell/guia4.hs
    \item Dada una lista de figuras xs expresa las siguientes propiedades utilizando los cuantificadores y los predicados y funciones ya definidas
    \begin{enumerate}
        \item \( \langle \forall x : x \in_\ell  \, xs : x.rojo  \rangle \)
        \item \( \langle \forall x : x \in_\ell  \, xs : x.tam < 5  \rangle \)
        \item \( \langle \forall x : x \in_\ell  \, xs : x.triangulo \land x.rojo  \rangle \)
        \item \( \langle \exists x : x \in_\ell  \, xs : x.cuadrado \land x.verde  \rangle \)
        \item \( \langle \forall x : x \in_\ell  \, xs \land x.circulo : x.azul \land x.tam < 10 \rangle \)
        \item \( \langle \nexists x : x \in_\ell  \, xs \land x.triangulo : x.azul  \rangle \)
        \item \( \langle \nexists x : x \in_\ell  \, xs \land x.triangulo : x.azul \lor x.verde \rangle \)
        \item \( \langle \exists x : x \in_\ell  \, xs : x.cuadrado \land x.tam < 5  \rangle \)
        \item \( \langle \exists x : x \in_\ell  \, xs : x.circulo \land x.rojo  \rangle \implies \langle \exists x \in_\ell xs : x.cuadrado \land x.rojo \rangle \) 
    \end{enumerate}

    \pagebreak

    \item Para cada propiedad del ejercicio 3 defini una función recursiva que dada una lista devuelva verdadero si
    la propiedad se cumple para esa lista y falso en caso contrario. Por ejemplo, para el predicado “Todas las
    figuras de xs son rojas” de la propiedad 3a

    \begin{enumerate}
        \item
        \( xs = [(\text{Triangulo}, \text{Rojo},10),(\text{Cuadrado}, \text{Rojo},20),(\text{Circulo}, \text{Rojo}, 20)] \) 
            \( xs' = [(\text{Cuadrado}, \text{Azul}, 10),(\text{Circulo}, \text{Rojo},40),(\text{Triangulo}, \text{Rojo},30)] \)
        \item
        \(  xs = [(\text{Cuadrado}, \text{Azul}, 3),(\text{Cuadrado}, \text{Rojo}, 4), (\text{Circulo}, \text{Amarillo}, 1)] \) 
            \(  xs' = [(\text{Cuadrado}, \text{Azul}, 3),(\text{Cuadrado}, \text{Rojo}, 4), (\text{Circulo}, \text{Amarillo}, 6)] \)
        \item
        \( xs = [(\text{Cuadrado}, \text{Azul}, 3),(\text{Triangulo}, \text{Rojo}, 4), (\text{Circulo}, \text{Amarillo}, 6)] \) 
            \( xs' = [(\text{Cuadrado}, \text{Azul}, 3),(\text{Triangulo}, \text{Rojo}, 4), (\text{Triangulo}, \text{Amarillo}, 6)] \)
        \item
        \( xs = [(\text{Cuadrado}, \text{Verde}, 3),(\text{Triangulo}, \text{Rojo}, 4), (\text{Circulo}, \text{Amarillo}, 6)]  \) 
            \( xs = [(\text{Cuadrado}, \text{Azul}, 3),(\text{Triangulo}, \text{Rojo}, 4), (\text{Circulo}, \text{Amarillo}, 6)] \)
        \item
        \( xs = [(\text{Circulo}, \text{Azul}, 3),(\text{Triangulo}, \text{Rojo}, 4), (\text{Circulo}, \text{Azul}, 6)]  \) \\
            \( xs' = [(\text{Cuadrado}, \text{Verde}, 3),(\text{Triangulo}, \text{Rojo}, 4), (\text{Circulo}, \text{Amarillo}, 6)] \)
        \item
        \(xs = [(\text{Cuadrado}, \text{Verde}, 3),(\text{Triangulo}, \text{Rojo}, 4), (\text{Triangulo}, \text{Amarillo}, 6)] \) 
        \( xs' = [(\text{Cuadrado}, \text{Verde}, 3),(\text{Triangulo}, \text{Azul}, 4), (\text{Circulo}, \text{Amarillo}, 6)] \)
        \item
        \( xs = [(\text{Cuadrado}, \text{Verde}, 3),(\text{Triangulo}, \text{Rojo}, 4), (\text{Circulo}, \text{Rojo}, 6)] \) 
        \(xs' = [(\text{Cuadrado}, \text{Verde}, 3),(\text{Triangulo}, \text{Rojo}, 4), (\text{Circulo}, \text{Amarillo}, 6)] \)
        \item
        \( xs = [(\text{Cuadrado}, \text{Verde}, 3),(\text{Triangulo}, \text{Rojo}, 4), (\text{Circulo}, \text{Amarillo}, 6)] \) 
            \( xs' = [(\text{Cuadrado}, \text{Verde}, 8),(\text{Triangulo}, \text{Rojo}, 4), (\text{Circulo}, \text{Amarillo}, 6)] \)
        \item
            \( xs = [(\text{Cuadrado}, \text{Verde}, 3),(\text{Triangulo}, \text{Rojo}, 4), (\text{Circulo}, \text{Amarillo}, 6)] \) 
    \end{enumerate}

    \item Realizados en FaMAF/Algoritmos 1/Haskell/guia4.hs
    \item Construi una lista de figuras xs en las que se satisfagan progresivamente cada una de las siguientes
    sentencias. Formalizá las oraciones con la lógica de predicados.
    \begin{enumerate}
        \item Alguna figura de xs es de tamaño mayor a 10. \\
        \( xs = [(\text{Cuadrado}, \text{Verde}, 3),(\text{Triangulo}, \text{Rojo}, 4), (\text{Circulo}, \text{Amarillo}, 6)] \) 
        \item Hay un cuadrado en xs. \\
        \( xs = [(\text{Cuadrado}, \text{Verde}, 3),(\text{Triangulo}, \text{Rojo}, 4), (\text{Circulo}, \text{Amarillo}, 6)] \)
    \end{enumerate}
\end{enumerate}

\end{document}
